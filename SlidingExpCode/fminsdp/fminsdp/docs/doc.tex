\documentclass{article}

\usepackage{amsmath}
\usepackage{mathrsfs}
\usepackage{amsfonts}
\usepackage{graphicx}
\usepackage{multirow}
\usepackage{alltt}
\usepackage[numbers]{natbib}
\usepackage[total={6in, 9in},top=1in,left=1.1in]{geometry}
\usepackage{fancyhdr}
\usepackage{hyperref}


\fancypagestyle{plain}{%
  \renewcommand{\headrulewidth}{0pt}%
  \fancyhf{}%
  \fancyfoot[C]{\footnotesize Page \thepage\ of \pageref{LastPage}}%
}

\lhead{Carl-Johan Thore}
\rhead{fminsdp - Optimization with matrix inequality constraints}
\pagestyle{fancyplain}


\newcommand{\bm}[1]{\mbox{\boldmath $#1$}}
\newcommand{\T}{\textsf{T}}
\newcommand{\svec}{\mbox{\textsf{svec}}}


\hypersetup{bookmarksopen=true}

\title{FMINSDP -- a code for solving optimization problems with matrix inequality constraints}
\date{\today}
\author{Carl-Johan Thore}

%%%%%%%%%%%%%%%%%%%%%%%%%%%%%%%%%%%s

\begin{document}

\maketitle

\thispagestyle{empty}

\noindent This document is a theoretical and practical introduction to the Matlab code \texttt{fminsdp} designed to find local solutions to (small-scale) non-linear, non-convex optimization problems (NLPs) with both scalar constraints and matrix inequality constraints. 
\vskip 2mm
%%%%%%%%%%%%%%%%%%%%%%%%%%%%%%%%%%%%%%%%%%%%%%%%%%%%%%%%%%%%%%
%%%%%%%%%%%%%%%%%%%%%%%%%%%%%%%%%%%%%%%%%%%%%%%%%%%%%%%%%%%%%%
%%%%%%%%%%%%%%%%%%%%%%%%%%%%%%%%%%%%%%%%%%%%%%%%%%%%%%%%%%%%%%

\noindent \textbf{Notation.} A matrix $\bm{A} \in \mathbb{R}^{m \times m}$ is said to be \textit{positive semi-definite} if 
$\bm{y}^{\T}\bm{A}\bm{y} \geq 0$ for all $\bm{y} \in \mathbb{R}^{m}$. It is convenient to introduce
the notation "$\bm{A} \succeq \bm{0}$" to indicate that $\bm{A}$ is positive semi-definite. A \textit{matrix inequality} is here
defined as an expression of the form
\begin{equation}\label{eq:matrix_ineq}
\bm{\mathcal{A}}(\bm{x}) \succeq \bm{0},
\end{equation}
where $\bm{\mathcal{A}}$ is a map from $\Omega \subset \mathbb{R}^{n}$ to the space of symmetric matrices of size
$m \times m$ with real entries, $\mathbb{S}^{m}$.


%%%%%%%%%%%%%%%%%%%%%%%%%%%%%%%%%%%%%%%%%%%%%%%%%%%%%%%%%%%%%%
%%%%%%%%%%%%%%%%%%%%%%%%%%%%%%%%%%%%%%%%%%%%%%%%%%%%%%%%%%%%%%
%%%%%%%%%%%%%%%%%%%%%%%%%%%%%%%%%%%%%%%%%%%%%%%%%%%%%%%%%%%%%%


\section{The optimization problem}

\texttt{fminsdp} attempts to find a local solution to non-linear, non-convex optimization problems of the form
\begin{equation}\label{eq:problem}
\quad
	\begin{aligned}
	&	\underset{\bm{x}\in\mathbb{R}^{n}}{\mbox{minimize}} \;\; f(\bm{x})  \\
	&	\mbox{subject to} \;
	\left\{
		\begin{aligned}
			& \bm{A}_{eq}\bm{x} = \bm{b}_{eq} & \mbox{linear equality constraints}      \\
			& \bm{A}\bm{x} \leq \bm{b}	      & \mbox{linear inequality constraints} \\
			& \bm{c}_{eq}(\bm{x}) = \bm{0}	 	& \mbox{nonlinear equality constraints}			\\		
		  & \bm{c}(\bm{x}) \leq \bm{0}			& \mbox{nonlinear inequality constraints}  \\			
			& \bm{l} \leq \bm{x} \leq \bm{u}	& \mbox{box constraints}\\
			& \bm{\mathcal{A}}_{i}(\bm{x}) \succeq \bm{0}, \quad i = 1,...,q & \mbox{matrix inequality constraints},
		\end{aligned}
		\right.
	\end{aligned}
\end{equation}
\vskip 2mm
\noindent where $\bm{A}_{eq}$, $\bm{A}$, $\bm{b}_{eq}$, $\bm{b}$, $\bm{l}$ and $\bm{u}$ are constant matrices and vectors, respectively. The functions $f$, $\bm{c}$,  $\bm{c}_{eq}$ and $\bm{\mathcal{A}}_{i}:\Omega_{i} \rightarrow \mathbb{S}^{m_{i}}$, $i = 1,...,q$, can be non-linear and are (preferably) at least 
twice continuously differentiable. Users familiar with \texttt{fmincon} from the Optimization Toolbox in Matlab \cite{fmincon:60} should recognize the 
form of problem \eqref{eq:problem} --- the novelty here is the addition of $q$ matrix inequality constraints. 

As it stands, problem \eqref{eq:problem} is not in a form that can be treated directly by currently available NLP-solvers. Therefore, \texttt{fminsdp} works with a reformulation of problem \eqref{eq:problem} described next.

%%%%%%%%%%%%%%%%%%%%%%%%%%%%%%%%%%%%%%%%%%%%%%%%%%%%%%%%%%%%%%
%%%%%%%%%%%%%%%%%%%%%%%%%%%%%%%%%%%%%%%%%%%%%%%%%%%%%%%%%%%%%%
%%%%%%%%%%%%%%%%%%%%%%%%%%%%%%%%%%%%%%%%%%%%%%%%%%%%%%%%%%%%%%


\section{Theoretical background}

To handle constraints of the form \eqref{eq:matrix_ineq}, \texttt{fminsdp} reformulates them into sets of scalar equality constraints based on the fact that a matrix is positive semi-definite if and only if it admits a Cholesky decomposition. In other words, 
\vskip 2mm
\noindent \textbf{Theorem 1.} A symmetric matrix $\bm{A} \succeq \bm{0}$ if and only if
\begin{equation}\nonumber
\bm{A} = \bm{L}\bm{L}^{\T}
\end{equation}
for some lower triangular matrix $\bm{L}$. 
\vskip 2mm
\noindent By default in \texttt{fminsdp}, $\bm{L}$ is restricted to the space of lower triangular $m \times m$-matrices with non-negative diagonal elements, referred to as $\mathcal{L}^{m}$. Non-negativity of the diagonal elements is not required by the theory, but enforcement of this condition is often important for computational efficiency.

\texttt{fminsdp} makes use of the function $\svec : \mathbb{R}^{m\times m} \rightarrow \mathbb{R}^{p}$, where $p$ denotes the number of  non-zero elements in the Cholesky factor $\bm{L}$ of a given matrix; i.e., the number of elements in the sparsity pattern of $\bm{L}$, obtained by a \textit{symbolic} Cholesky factorization of the given matrix. $\svec$ takes, column-wise, the elements of the input matrix corresponding to potential non-zeros of $\bm{L}$ and stacks them on top of each other to form a vector of length $p$. If the lower triangular part of $\bm{L}\in\mathcal{L}^{m}$ is full we have $p=m(m+1)/2$ and
\begin{equation}\nonumber
\svec(\bm{A}) = (A_{11},A_{21},\ldots,A_{m1},A_{22},\ldots,A_{m2},\ldots,A_{mm})^{\T};
\end{equation}
i.e., the vector $\svec(\bm{A})$ contains the elements of the lower triangular part of $\bm{A}$. 

Now, based on Theorem 1, problem \eqref{eq:problem} is reformulated into the following problem:
\begin{equation}\label{eq:problem2}
\quad
	\begin{aligned}
	&	\underset{\bm{x}\in\mathbb{R}^{n},\; \bm{\ell}_{1}\in\mathbb{R}^{p_{1}},\;\ldots ,\;\bm{\ell}_{q}\in\mathbb{R}^{p_{q}}}{\mbox{minimize}} \;\; f(\bm{x})  \\
	&	\mbox{subject to} \;
	\left\{
		\begin{aligned}
			& \bm{A}_{eq}\bm{x} = \bm{b}_{eq} &\\
			& \bm{A}\bm{x} \leq \bm{b}	      &\\
			& \bm{c}_{eq}(\bm{x}) = \bm{0}	 	&\\
		  & \bm{c}(\bm{x}) \leq \bm{0}			&\\
			& \bm{l} \leq \bm{x} \leq \bm{u}	&\\
			& \tilde{\bm{l}} \leq (\bm{\ell}_{1},\ldots,\bm{\ell}_{q}) \leq \tilde{\bm{u}} & \\
			& \svec\left(\bm{\mathcal{A}}_{i}(\bm{x}) - \bm{L}_{i}(\bm{\ell}_{i})\bm{L}_{i}(\bm{\ell}_{i})^{\T}\right) = \bm{0}, & \quad i = 1,...,q \\
			& \mbox{\textsf{diag}}\{\bm{L}_{i}(\bm{\ell}_{i})\} \geq \bm{0}, & \quad i = 1,...,q,
		\end{aligned}
		\right.
	\end{aligned}
\end{equation}
where the variables $\bm{\ell}_{i}$ defining the non-zero elements of the Cholesky factors are referred to as \textit{auxiliary variables}, $\tilde{\bm{l}}$ are $\tilde{\bm{u}}$ are constant vectors, and $\mbox{\textsf{diag}}$ returns a vector containing the diagonal elements of a matrix. This problem is in a form amenable to direct treatment by an NLP-solver, and this is the problem to which \texttt{fminsdp} attempts to find a local solution.

\vskip 2mm
\noindent \textbf{Note.} Any local minimum of problem \eqref{eq:problem2} is also a local minimum for \eqref{eq:problem}, and vice versa (assuming that bounds on the auxiliary variables do not prevent this). However, problem \eqref{eq:problem2} may have additional stationary points (KKT-points) not present in \eqref{eq:problem}. The author has not experienced any difficulties obviously attributed to this fact, but it cannot be ruled out that it might cause trouble on some problems.
\vskip 2mm

A key motivation behind \texttt{fminsdp} is to abstract away the exact treatment of the matrix inequality constraints so that the user only ''sees'' a problem of the form \eqref{eq:problem}; that is, the user should not have to deal with the Cholesky factors and the auxiliary variables. Thus, when using \texttt{fminsdp} one only works with the primary variables $\bm{x}$, and user-supplied derivatives are only with respect to $\bm{x}$. The examples in Section \ref{sec:tutorial} and the folder \texttt{examples} shows how this is done in practice.

%%%%%%%%%%%%%%%%%%%%%%%%%%%%%%%%%%%%%%%%%%%%%%%%%%%%%%%%%%%%%%%%%%%%%%%%%%%%%%%%%%

\subsection{Problem sizes}

It was mentioned, parenthetically, in the first line of this document that \texttt{fminsdp} seeks local solutions to \textit{small-scale} problems. If the constraint matrices are of small sizes or/and very sparse, then \texttt{fminsdp}
might in fact be able to solve large-scale problems within reasonable time. In general however, one can expect a fairly large
number of auxiliary variables and thus many non-zero elements in the constraint Jacobian (and Hessian of the Lagrangian), resulting
in much memory and CPU time being devoted to solution of linear systems by the NLP solver used to solve \eqref{eq:problem2}.

 
%%%%%%%%%%%%%%%%%%%%%%%%%%%%%%%%%%%%%%%%%%%%%%%%%%%%%%%%%%%%%%%%%%%%%%%%%%%%%%%%%%
%%%%%%%%%%%%%%%%%%%%%%%%%%%%%%%%%%%%%%%%%%%%%%%%%%%%%%%%%%%%%%%%%%%%%%%%%%%%%%%%%%
%%%%%%%%%%%%%%%%%%%%%%%%%%%%%%%%%%%%%%%%%%%%%%%%%%%%%%%%%%%%%%%%%%%%%%%%%%%%%%%%%%


\subsection{Infeasible initial points}

It is recommended (but not necessary!) that the user supply an initial point which is feasible with respect to the constraints of problem \eqref{eq:problem}, in particular the matrix inequality constraints. If the latter is not possible, an option can be set (see section \ref{sec:using}) so that \texttt{fminsdp} attempts to solve the following problem in place of \eqref{eq:problem2}:
\begin{equation}\label{eq:problem3}
\quad
	\begin{aligned}
	&	\underset{\bm{x}\in\mathbb{R}^{n},\; \bm{\ell}_{1}\in\mathbb{R}^{p_{1}},\;\ldots ,\;\bm{\ell}_{q}\in\mathbb{R}^{p_{q}},\;\mbox{$s$}\in\mathbb{R}}{\mbox{minimize}} \;\; f(\bm{x}) + cs \\
	&	\mbox{subject to} \;
	\left\{
		\begin{aligned}
			& \bm{A}_{eq}\bm{x} = \bm{b}_{eq} &\\
			& \bm{A}\bm{x} \leq \bm{b}	      &\\
			& \bm{c}_{eq}(\bm{x}) = \bm{0}	 	&\\
		  & \bm{c}(\bm{x}) \leq \bm{0}			&\\
			& \bm{l} \leq \bm{x} \leq \bm{u}	&\\
			& \tilde{\bm{l}} \leq (\bm{\ell}_{1},\ldots,\bm{\ell}_{q}) \leq \tilde{\bm{u}} & \\
			& \svec\left(\bm{\mathcal{A}}_{i}(\bm{x}) + s\bm{I}^{m_{i}} - \bm{L}_{i}(\bm{\ell}_{i})\bm{L}_{i}(\bm{\ell}_{i})^{\T}\right) = \bm{0}, &\quad i = 1,...,q \\
			& \mbox{\textsf{diag}}\{\bm{L}_{i}(\bm{\ell}_{i})\} \geq \bm{0}, & \quad i = 1,...,q \\
			& \underline{s} \leq s \leq \overline{s}. & 
		\end{aligned}
		\right.
	\end{aligned}
\end{equation}
Here $s$ is an auxiliary variable, $\bm{I}^{m_{i}},\, i = 1,...,q$, are identity matrices of size $m_{i} \times m_{i}$, and
$\underline{s}$ and $\overline{s}$ are constants. The constant $c$ appearing in the objective should be set to some large positive number (finding a suitable value might require experimenting a bit) such that $s$ becomes close to zero at a solution.


%%%%%%%%%%%%%%%%%%%%%%%%%%%%%%%%%%%%%%%%%%%%%%%%%%%%%%%%%%%%%%
%%%%%%%%%%%%%%%%%%%%%%%%%%%%%%%%%%%%%%%%%%%%%%%%%%%%%%%%%%%%%%
%%%%%%%%%%%%%%%%%%%%%%%%%%%%%%%%%%%%%%%%%%%%%%%%%%%%%%%%%%%%%%


\section{Alternatives}

There are many important special cases of problem \eqref{eq:problem} for which efficient specialized solvers 
are available. Although this is possible, \texttt{fminsdp} is not intended to replace such solvers, which, 
when applicable, can be orders of magnitude faster.

The following is an incomplete list of currently available solvers:
\begin{itemize}
\item Problems with linear matrix inequality (LMI) constraints:\\
\hskip2mm SeDuMi, SDPT3, SDPA, LMI Lab, PENSDP, BMISolver, PENBMI, PENNON
\item Problems with bilinear matrix inequality (BMI) constraints:.\\
\hskip2mm BMISolver, PENBMI, PENNON
\item General problems of type \eqref{eq:problem}:\\
\hskip2mm PENNON
\end{itemize}
(Except for PENNON, there are additional constraints on the structure of the problems treated by the listed solvers). Note that PENNON \cite{Kocvara:2003} (and its open-source version PENLAB) handles the same class of problems as \texttt{fminsdp} and may therefore be considered as an alternative to this solver.

The Matlab code YALMIP \cite{Yalmip:2004} provides a convenient unified interface to the solvers listed above (except BMISolver).

One could of think of other criteria, not without drawbacks of course, for positive semi-definiteness besides the Cholesky factorization that might
be used to obtain a standard NLP formulation of \eqref{eq:problem}. Non-negativity of the smallest eigenvalue or of the leading principle minors of a matrix are both necessary and sufficient criteria. If one is satisfied with sufficiency, diagonal dominance could also be considered.


%%%%%%%%%%%%%%%%%%%%%%%%%%%%%%%%%%%%%%%%%%%%%%%%%%%%%%%%%%%%%
%%%%%%%%%%%%%%%%%%%%%%%%%%%%%%%%%%%%%%%%%%%%%%%%%%%%%%%%%%%%%%
%%%%%%%%%%%%%%%%%%%%%%%%%%%%%%%%%%%%%%%%%%%%%%%%%%%%%%%%%%%%%%



\section{Using \texttt{fminsdp}}
\label{sec:using}

A user of \texttt{fminsdp} is expected to provide two functions: one for evaluating the objective function (1) and one for evaluating the non-linear constraints (2).
\begin{enumerate}
\item \texttt{[fval,\textit{grad}] = objfun(x)}
\item \texttt{[cineq,ceq,\textit{cineqgrad},\textit{ceqgrad}] = nonlcon(x)}
\end{enumerate}
(Here and in the following it is assumed that the reader is familiar with the Matlab syntax.) The return arguments written in italics are optional and only needed if the user wishes to provide gradients for the objective and constraints. The matrix inequality constraints are defined in the non-linear constraints function (even linear matrix inequalities). The values of the constraint matrices, vectorized using \svec, are returned as the last elements in the output vector \texttt{ceq}; see Section \ref{sec:tutorial} for an example.

In addition to objective and non-linear constraints functions, the user may optionally provide a function for evaluating
the Hessian of the Lagrangian:
\begin{itemize}
\item \texttt{H = hessian(x,lambda)},
\end{itemize}
where \texttt{lambda} is a struct with two fields \texttt{ineqnonlin} and \texttt{eqnonlin} containing values of the Lagrange
multipliers associated with the non-linear inequality and equality constraints, respectively, and one field \texttt{sigma} 
which only used when running with NLP-solver Ipopt.

The calling syntax of \texttt{fminsdp} is similar to that of \texttt{fmincon}:
\begin{verbatim}
>> [x,fval,exitflag,output,lambda,grad,hessian] = ...
										                        fminsdp(objfun,x0,A,b,Aeq,beq,lb,ub,nonlcon,options)
\end{verbatim}
Unlike fmincon, at least 9 input arguments are required. If your problem has no linear constraints and no 
bounds, the corresponding input arguments can be set to empty matrices. Note that \texttt{fminsdp} does not
support passing additional arguments to the objective or constraint functions through an 11-th input argument. 
To pass additional arguments to the user-supplied functions one should use anonymous or nested functions.

\subsection{Options}
A number of options are available when calling \texttt{fminsdp}. These are passed to the code in the form of a struct containing
parameter-value pairs. The most convenient way to specify options is to use the function \texttt{sdpoptionset}:
\begin{verbatim}
>> options = sdpoptionset('MatrixInequalities','on');
\end{verbatim}

In addition to the options accepted by the function \texttt{optimset} from the Matlab Optimization Toolbox, the
following options are available when calling \texttt{fminsdp}:
\vskip 2mm
\noindent \texttt{MatrixInequalities}  \hskip 1cm logical scalar\\
Indicates whether or not the problem has any matrix inequality constraints. If not, \texttt{fminsdp} will simply call the 
NLP solver specified using options.NLPsolver.\\
\noindent  Default: true 
\vskip 2mm
\noindent \texttt{Aind}  \hskip 3.5cm scalar or numeric array\\
Marks the beginning of each matrix constraint in the vector \texttt{ceq} returned from the non-linear constraints function. If
the option \texttt{sp\_pattern}, described below, is used it is only necessary to mark the beginning of the \texttt{first} matrix 
constraint.\\
\noindent  Default: 1
\vskip 2mm
\noindent \texttt{NLPsolver} \hskip 2.5cm  \{'fmincon', 'ipopt', 'snopt', 'knitro', 'mma', 'gcmma'\}               \\
Select NLP-solver for the reformulated problem. Currently \texttt{fminsdp} provides interfaces to fmincon, Ipopt \cite{Wachter:2006}, SNOPT \cite{Gill:2002}, KNITRO \cite{Byrd:2006}, MMA/GCMMA \cite{Svanberg:2007}.
NOTE: If you run Ipopt older than 3.11.0, make sure to modify the file ipopt\_main.m appropriately. \\
Default: 'fmincon'
\vskip 2mm
\noindent \texttt{sp\_pattern}  \hskip 2.3cm  (sparse) matrix or cell array of matrices\\
Sparsity patterns of the Cholesky factors of the matrix constraints. These can be generated by a call to \texttt{symbol\_fact}.  
If this option is used, one must provide one matrix for each matrix constraint. 
\\
\noindent  Default: []
\vskip 2mm
\noindent \texttt{L0} \hskip 3.2cm  (sparse) matrix  or cell array of matrices \\
Initial guess for the Cholesky factors of the matrix constraints. If the user does not provide an initial
guess, \texttt{fminsdp} will generate one by attempting a Cholesky factorization of the constraint matrices 
at the initial point. If this fails, \texttt{fminsdp} will add a multiple of the identity matrix to
the constraint matrices until all of them are positive definite and use the Cholesky factorizations of these
matrices as an initial guess. \\
Default: []
\vskip 2mm
\noindent \texttt{Ldiag\_low} \hskip 2cm  scalar or numeric array	\\
Lower bound(s) on the diagonal elements of the Cholesky factors. \\
Default: 0
\vskip 2mm
\noindent \texttt{L\_low} \hskip 2.75cm   scalar or array of doubles \\
Lower bound(s) on the off-diagonal element of the Cholesky factors \\
Default: -inf
\vskip 2mm
\noindent 
\texttt{L\_upp} \hskip 2.75cm  scalar or array of doubles \\
Upper bound(s) on the elements of the Cholesky factors (including the diagonal elements).\\
Default: inf
\vskip 2mm
\noindent 
\texttt{c} \hskip 3.47cm  non-negative scalar    \\
By setting $c>0$ \texttt{fminsdp} will attempt to solve problem \eqref{eq:problem3} instead of 
\eqref{eq:problem2}. The user may in this case also let the objective function be empty; i.e., the first input argument to \texttt{fminsdp} can be set to '[]'. This is useful when one wants to check feasibility of one or more matrix inequalities. \\
Default: 0
\vskip 2mm
\noindent
\texttt{s\_low}   \hskip 2.8cm        scalar      \\
Lower bound on the auxiliary variable $s$ in problem \eqref{eq:problem3}.\\
Default: 0
\vskip 2mm
\noindent
\texttt{s\_upp}   \hskip 2.8cm        scalar      \\
Upper bound on the auxiliary variable $s$ in problem \eqref{eq:problem3}.\\
Default: inf
\vskip 2mm
\noindent
\texttt{HessianCheck} \hskip 1.4cm   \{'on', 'off'\} \\      
Simple check of Hessian of the Lagrangian against finite differences at the initial point. 
Assumes you have the code DERIVEST, which must be obtained separately, on your Matlab path. 
This check can be very time consuming and should preferably be carried out on small instances of 
a problem.\\
Default: 'off'
\vskip 2mm
\noindent
\texttt{HessMult} \hskip 2cm   \{function\_handle, 'on'\} \\
If using fmincon with \texttt{options.SubProblemAlgorithm = 'cg'}, you can work with Hessian 
times vector products directly, thereby avoiding the formation of the full Hessian of
the Lagrangian. Set to a function handle or simply to 'on' if the Hessian of
the Lagrangian with respect to the primary variables is zero. \\
Default: []
\vskip 2mm
\noindent
\texttt{ipopt} \hskip 2.7cm     struct   \\
Options to be passed on to NLP solver Ipopt. Please refer to the Ipopt documentation for a
list of available options. \\
Default: []
\vskip 2mm
\noindent
\texttt{eigs\_opts} \hskip 2.1cm       struct    \\
Options passed to the Matlab function \texttt{eigs} used for eigenvalue computations. \\
Default: struct('isreal',true,'issym',true)
\vskip 2mm
\noindent
\texttt{KnitroOptionsFile} \hskip 0.5cm  character array \\
Name of an options file to be read by NLP solver KNITRO. Please refer to the KNITRO
documentation for a list of available options.\\
Default: []
\vskip 2mm
\noindent
\texttt{SnoptOptionsFile} \hskip 0.6cm   character array \\
Name of an options file to be read by NLP solver SNOPT. Please refer to the SNOPT
documentation for a list of available options.\\
Default: []
\vskip 2mm
\noindent
\texttt{GradPattern} \hskip 1.5cm        numeric array   \\
Sparsity pattern for the gradient of the objective function. Only used by NLP solver SNOPT and 
only effective if you also set \texttt{options.JacobPattern} (see the fmincon documentation for
details on the latter). \\
Default: []

%%%%%%%%%%%%%%%%%%%%%%%%%%%%%%%%%%%%%%%%%%%%%%%%%%%%%%%%%%%%%%%%%%%%%%%%%%%%


\subsection{Output}

The available output arguments from \texttt{fminsdp} are the same as those of fmincon:
\begin{verbatim}
>> [x,fval,exitflag,output,lambda,grad,hessian] = fminsdp(...)
\end{verbatim}
The only difference is that the struct \texttt{output}, the fourth output argument, contains
some additional fields:
\vskip 3mm
\noindent
\texttt{A} \hskip 4.2cm        cell array of matrices  \\
Constraint matrices evaluated at the solution \texttt{x}. 
\vskip 2mm
\noindent
\texttt{L} \hskip 4.2cm        cell array of matrices  \\
Cholesky factors of the constraint matrices evaluated at the solution \texttt{x}. These are computed
by assembling each $\bm{L}_{i}$ from the variable vector $\bm{\ell}_{i}$ and not by a Cholesky factorization
of the corresponding constraint matrix.
\vskip 2mm
\noindent
\texttt{L0} \hskip 4.0cm        cell array of matrices  \\
Cholesky factors of the constraint matrices evaluated at the initial point \texttt{x0}.
\vskip 2mm
\noindent \texttt{nxvars}		\hskip 3.2cm 		numeric scalar \\
Number of primary variables.
\vskip 2mm
\noindent \texttt{nLvars}		\hskip 3.2cm     numeric scalar \\
Number of auxiliary variables.
\vskip 2mm
\noindent \texttt{A\_size}   \hskip 3.3cm     numeric array \\
Size of the constraint matrices; i.e., $m_{i}$ in \eqref{eq:problem2}.
\vskip 2mm
\noindent \texttt{nMatrixConstraints}   \hskip 1cm     numeric scalar \\
Number of matrix constraints.
\vskip 2mm
\noindent \texttt{NLPsolver}   \hskip 2.7cm     string \\
Selected NLP solver.
\vskip 3mm
It the user has set \texttt{options.c} to some positive number in order to solve problem \eqref{eq:problem3},
then two additional fields are available:
\vskip 3mm
\noindent \texttt{s0}   \hskip 2.7cm     double scalar \\
Initial value for the auxiliary variable $s$.
\vskip 2mm
\noindent \texttt{s}   \hskip 2.7cm     double scalar \\
Value of $s$ at the solution.


%%%%%%%%%%%%%%%%%%%%%%%%%%%%%%%%%%%%%%%%%%%%%%%%%%%%%%%%%%%%%%%%%%%%%%%%%%%%
%%%%%%%%%%%%%%%%%%%%%%%%%%%%%%%%%%%%%%%%%%%%%%%%%%%%%%%%%%%%%%%%%%%%%%%%%%%%
%%%%%%%%%%%%%%%%%%%%%%%%%%%%%%%%%%%%%%%%%%%%%%%%%%%%%%%%%%%%%%%%%%%%%%%%%%%%


\section{A tutorial example}
\label{sec:tutorial}

To aid the user, a very simple tutorial example is provided here. For more details, please refer to the examples 
found in the \texttt{examples}-folder. 

Consider the following (non-sense) problem:
\begin{equation}\label{eq:tutorial}
\quad
	\begin{aligned}
    & \underset{\bm{x}\in\mathbb{R}}{\mbox{minimize}} \;\;  x^{2}  \\
	& \mbox{subject to} \;
	\left\{
		\begin{aligned}
			& x^{3} = 0 	\\
		    & x^{2} \leq 0			\\
			& \left(
			\begin{array}{cc}
					x   & x^2 \\
			        x^2 & 0
			      \end{array}
			\right) \succeq \bm{0}		
		\end{aligned}
		\right.
	\end{aligned}
\end{equation}
To solve this problem using \texttt{fminsdp} we need to implement at least two functions:

\begin{enumerate} 
\item The objective function:
\begin{verbatim}
function [fval,grad] = objfun(x)
fval = x^2;
if nargout>1 
   grad = 2*x;
end
\end{verbatim}

\item The non-linear constraints function:
\begin{verbatim}
function [cineq,ceq,cineqgrad,ceqgrad] = nonlcon(x)
cineq = x^2;
ceq = [x^3; svec([x x^2; x^2 0]))];
if nargout>2
   cineqgrad = 2*x;
   ceqgrad = [3*x^2 svec([1 2*x; 2*x 0])'];
end
\end{verbatim}

\end{enumerate}

\noindent A function for evaluating the Hessian of the Lagrangian is optional, but recommended. One
such function is given here:
\begin{verbatim}
function H = hessian(x, lambda)
H = lambda.ineqnonlin*2 + lambda.eqnonlin(1)*6*x  + ...
    lambda.eqnonlin(2:end,1)'*svec([0 2; 2 0]));
\end{verbatim}

Using the functions specified above, a simple script to solve problem \eqref{eq:tutorial} can now be written:

\begin{verbatim}

% Mark the beginning of the matrix inequality constraints in the vector ceq 
% returned from nonlcon
options.Aind = 2;

% Specify that analytical gradients should be used
options.GradObj = 'on';
options.GradConstr = 'on';

% Specify that the function "hessian" should be used for the
% Hessian of the Lagrangian
options.Hessian = 'user-supplied'
options.HessFcn = @(x,lambda) hessian(x,lambda);

% Specify initial point
x0 = 1;

% Call fminsdp
[x,fval] = fminsdp(objfun,x0,[],[],[],[],[],[],nonlcon,options);
				   
\end{verbatim}	


%%%%%%%%%%%%%%%%%%%%%%%%%%%%%%%%%%%%%%%%%%%%%%%%%%%%%%%%%%%%%%%%%%%%%%%%%%%%
%%%%%%%%%%%%%%%%%%%%%%%%%%%%%%%%%%%%%%%%%%%%%%%%%%%%%%%%%%%%%%%%%%%%%%%%%%%%
%%%%%%%%%%%%%%%%%%%%%%%%%%%%%%%%%%%%%%%%%%%%%%%%%%%%%%%%%%%%%%%%%%%%%%%%%%%%



\section{Practicalities}
\label{sec:practicalities}

For a given problem there are usually a number of things that can be done to improve numerical performance. One can, for instance, try out various scalings or introduce new variables and constraints to reduce the degree of non-linearity. Here are some additional tips (the first one obviously specific to \texttt{fminsdp}): 

\vskip 5mm
\noindent\textbf{Exploit sparsity}
\vskip 2mm
\noindent In many cases, the functions $\bm{\mathcal{A}}_{i}$, $i = 1,...,q$ are sparse in the sense that the matrix $\bm{\mathcal{A}}_{i}(\bm{x})$
is sparse for every $\bm{x}$. This fact can be exploited to reduce the number of auxiliary variables and nonlinear constraints. 

To take advantage of sparsity, the user should compute the sparsity pattern of the Cholesky factor for each matrix constraint (or
an (pessimistic) estimate thereof) and supply this to \texttt{fminsdp} via the option \texttt{sp\_pattern}. For the
dummy problem in Section \ref{sec:tutorial} we simply add three lines to the driver script:
\begin{verbatim}
A = [1 1; 1 0]					  		        % Sparsity pattern of the constraint matrix
sp = symbol_fact(A)       % Symbolic Cholesky factorization
options.sp_pattern = sp;

...

[x,fval] = fminsdp(@(x) objfun(x),x0,[],[],[],...
				   [],[],[],@(x) nonlcon(x),options);

\end{verbatim}


\vskip 2mm
\noindent\textbf{Add artificial upper and lower bounds}
\vskip 2mm
\noindent Even though some of the variables in your problem has no  ''natural'' bounds on them, it is usually wise
to add upper and lower bounds. These bounds should of course be loose enough to not exclude any solution of 
interest, but even if they don't, setting them too tight might have a negative impact on the solution process.
Therefore, a bit of experimenting is recommended.

Bounds on the auxiliary variables can be specified by setting \texttt{options.L\_low} and \texttt{options.L\_upp}.


\vskip 5mm
\noindent\textbf{Experiment with various solvers and algorihms}
\vskip 2mm
\noindent \texttt{fminsdp} can use fmincon, SNOPT, KNITRO and Ipopt to solve problem \eqref{eq:problem2}. In addition, fmincon
and KNITRO lets the user choose between three different algorithms (two interior-point and one sqp). It is unlikely that a single solver and/or algorithm is best suited for all types of problems so it is recommended that the user experiment with various 
alternatives. The author's experience is that the interior-point method of fmincon (selected by setting \texttt{options.Algorithm='interior-point'} and \texttt{options.SubProblemAlgorithm = 'ldl-factorization'}) works well in
terms of the number of function evaluations, but that more efficient handling of the linear algebra makes the other solvers 
better suited for larger problems.

\vskip 5mm
\noindent\textbf{Provide gradients and Hessian of the Lagrangian}
\vskip 2mm
\noindent Providing code that computes gradients and the Hessian of the Lagrangian is usually a good idea (if you use Ipopt you \textit{must} provide code for evaluating gradients; SNOPT does not make use of code for evaluating the Hessian of the Lagrangian). Due to the homogeneity of $\svec$ one has simply
\begin{equation}\nonumber
\frac{\partial \svec(\bm{\mathcal{A}})}{\partial x_{i}} = \svec\left(\frac{\partial \bm{\mathcal{A}}}{\partial x_{i}}\right),
\end{equation}
and for a function $L(\bm{x}) = \bm{\lambda}^{\T}\svec(\bm{\mathcal{A}}(\bm{x}))$,
\begin{equation}\nonumber
\frac{\partial^2 L}{\partial x_{i}\partial x_{j}} =  \bm{\lambda}^{T}\svec\left(\frac{\partial^2 \bm{\mathcal{A}}}{\partial x_{i}\partial x_{j}}\right).
\end{equation}

The user is strongly recommended to check the correctness of the derivatives by comparing against finite-difference 
approximations. This can be done by setting \texttt{options.DerivativeCheck='on'} and \texttt{options.HessianCheck='on'}. 
If necessary, the accuracy of the finite-difference approximations in \texttt{fmincon} can be increased by setting
\texttt{options.FinDiffType='central'}.

If you use NLP solvers SNOPT, Ipopt or KNITRO, you can also provide sparsity patterns for the constraint Jacobian, and Ipopt
and KNITRO can also exploit a sparsity pattern for the Hessian of the Lagrangian. Said patterns are passed to the solvers
by setting \texttt{options.JacobPattern} and \\\texttt{options.HessPattern} to sparse matrices.

As an alternative to deriving and implementing derivatives one might consider using automatic differentiation, which provides 
derivatives with accuracy to machine precision, and, at least in principle, does so completely automatically.

			   

%%%%%%%%%%%%%%%%%%%%%%%%%%%%%%%%%%%%%%%%%%%%%%%%%%%%%%%%%%%%%%%%%%%%%%%%%%%%
%%%%%%%%%%%%%%%%%%%%%%%%%%%%%%%%%%%%%%%%%%%%%%%%%%%%%%%%%%%%%%%%%%%%%%%%%%%%
%%%%%%%%%%%%%%%%%%%%%%%%%%%%%%%%%%%%%%%%%%%%%%%%%%%%%%%%%%%%%%%%%%%%%%%%%%%%

\section{Examples}

Implementations of three example problem can be found in the folder \texttt{examples}. This section comprise a brief description of these examples, all of which are concerned with structural optimization of (plane) trusses. Please note that there are often many different ways to formulate such problems \cite{Cristensen:2009,Bendsoe:1994}, and formulations other than those given here may certainly be better suited to one's needs. Here, however, we are specifically interested in solving problems with matrix inequality constraints. 

Let $n_{d} = 2$ be the number of spatial dimensions, $N$ the number of nodes in the truss, and $n_{fixed}$ the number of prescribed (to zero) displacement components. The number of displacement degrees of freedom is $n=n_{d}N-n_{fixed}$ and there are $m$ potential bars in the truss.
The $m$ bar volumes are collected in a vector $\bm{x}\geq\bm{0}$ which is used to parametrize the design of the truss. Assuming (infinitesimally) small deformations and a quasi-static situation, the nodal displacement vector $\bm{u}\in\mathbb{R}^{n}$ satisfies the equilibrium equation
\begin{equation}\label{eq:equlibrium}
\bm{K}(\bm{x})\bm{u} = \bm{f},
\end{equation}
where $\bm{K}(\bm{x})$ is known as the stiffness matrix and $\bm{f}\in\mathbb{R}^{n}$ contains forces applied to the nodes. The  stiffness matrix is given by
\begin{equation}\nonumber
\bm{K}(\bm{x}) = \sum_{i=1}^{m}x_{i}E_{i}\bm{b}_{i}\bm{b}_{i}^{\T},
\end{equation}
where $x_{i}$ is the volume of the $i$:th bar, $E_{i}$ is Young's modulus of said bar, and the vector $\bm{b}_{i}$ depends on the geometry of the undeformed truss. Clearly, $\bm{K}(\bm{x})$ is positive semi-definite and symmetric. Assuming rigid body motions are prevented, by appropriate support conditions, and $\bm{x}>\bm{0}$, it is even positive definite.
\vskip2mm
\noindent\textbf{1.} The first problem in the \texttt{examples}-folder is to minimize the volume of a truss subject to the equilibrium condition \eqref{eq:equlibrium} and an upper bound $c$ on the compliance $\bm{f}^{\T}\bm{u}$. This can be formulated as a problem involving a single LMI: 
\begin{equation}\label{eq:prob1}
\quad
	\begin{aligned}
	&	\underset{\bm{x}\in\mathbb{R}^{m}}{\mbox{minimize}} \;\; \sum_{i=1}^{m}x_{i} \\
	&	\mbox{subject to} \;
	\left\{
		\begin{aligned}
		  & \left(
\begin{array}{cc}
c &\bm{f}^{\T}         \\
\bm{f} & \bm{K}(\bm{x})
\end{array}
\right) \succeq \bm{0} \\
			& x_{i} \geq 0 , \quad i=1,\ldots,m.
		\end{aligned}
		\right.
	\end{aligned}
\end{equation}

\vskip2mm
\noindent\textbf{2.} 
A drawback of problem \eqref{eq:prob1} is that the optimized structures may be unstable in the sense that they are prone to global buckling. One way to avoid many such designs is to impose an additional constraint \cite{Kocvara:2002} requiring that
\begin{equation}\label{eq:stability}
\bm{K}(\bm{x}) + \bm{G}(\bm{u}(\bm{x}),\bm{x}) \succeq \bm{0},
\end{equation}
where, the geometrix stiffness matrix, 
\begin{equation}\nonumber
\bm{G}(\bm{u}(\bm{x}),\bm{x}) = \sum_{i=1}^{m}x_{i}E_{i}\bm{b}_{i}^{\T}\bm{u}(\bm{x})\bm{\gamma}_{i}\bm{\gamma}_{i}^{\T},
\end{equation}
in which $\bm{u}(\bm{x})$ denotes a solution to \eqref{eq:equlibrium} and $\bm{\gamma}_{i}$, $i=1,\ldots,m$, depend on the geometry of the 
undeformed truss. 

Adding \eqref{eq:stability} to \eqref{eq:prob1} leads to the following problem involving both a linear and a non-linear matrix 
inequality\footnote{In practice it is perhaps better to replace the LMI in \eqref{eq:prob2} by the non-linear constraint $\bm{f}^{\T}\bm{u}(\bm{x})\leq c$ but we keep it to illustrate solution of a problem with more than one matrix inequality.}:
\begin{equation}\label{eq:prob2}
\quad
	\begin{aligned}
	&	\underset{\bm{x}\in\mathbb{R}^{m}}{\mbox{minimize}} \;\; \sum_{i=1}^{m}x_{i} \\
	&	\mbox{subject to} \;
	\left\{
		\begin{aligned}
		  & \left(
\begin{array}{cc}
c &\bm{f}^{\T}         \\
\bm{f} & \bm{K}(\bm{x})
\end{array}
\right) \succeq \bm{0} \\
			& \bm{K}(\bm{x}) + \bm{G}(\bm{u}(\bm{x}),\bm{x}) \succeq \bm{0} \\
			& x_{i} \geq \epsilon , \quad i=1,\ldots,m,
		\end{aligned}
		\right.
	\end{aligned}
\end{equation}
where $\epsilon$ is a small positive number introduced to avoid singularity of the stiffness matrix. \citet{Kocvara:2002} showed that a solution to this problem corresponds to a truss that will not exhibit global, linear buckling for loads
of the form $\tau\bm{f}$, $\tau\in[0,1)$.

\vskip2mm
\noindent\textbf{3.} 
Finally, the third example problem is an alternative formulation of problem \eqref{eq:prob2} obtained by treating the displacements as explicit variables in the optimization problem:
\begin{equation}\nonumber
\quad
	\begin{aligned}
	&	\underset{\bm{x}\in\mathbb{R}^{m},\bm{u}\in\mathbb{R}^{n}}{\mbox{minimize}} \;\; \sum_{i=1}^{m}x_{i} \\
	&	\mbox{subject to} \;
	\left\{
		\begin{aligned}
		  & \bm{f}^{\T}\bm{u} \leq c \\
		  & \bm{K}(\bm{x})\bm{u} = \bm{f} \\
			& \bm{K}(\bm{x}) + \bm{G}(\bm{u},\bm{x}) \succeq \bm{0} \\
			& x_{i} \geq \epsilon , \quad i=1,\ldots,m.
		\end{aligned}
		\right.
	\end{aligned}
\end{equation}
In this formulation, the upper bound on the compliance is a linear constraint, the equilibrium equation a set of bilinear equality constraints,
and the stability constraint is a BMI.

\vskip 5mm
\noindent To see how these problems can be solved with \texttt{fminsdp}, and what optimized designs might look like, please check out the Matlab codes found in the \texttt{examples}-folder.


\newpage
\bibliographystyle{../../../spbasic}
\bibliography{references}

\end{document}
